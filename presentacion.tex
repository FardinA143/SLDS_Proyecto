\documentclass{beamer}
\usepackage[utf8]{inputenc}
\usepackage[T1]{fontenc}
\usepackage[spanish]{babel}
% other packages
\usepackage{latexsym,amsmath,xcolor,multicol,booktabs,calligra}
\usepackage{graphicx,pstricks,listings,stackengine}

\author{Berta Ferré, Hajweria Hussain, Marc Ribas, Fardin Arafat, Francisco Ruiz}
\title{Software Libre en Dispositivos Móviles}
\subtitle{Sistemas operativos, impacto y desarrollo móvil}
\institute{Software Libre y Desarollo Social}
\date{\today}
\usepackage{CNU}

% defs
\def\cmd#1{\texttt{\color{red}\footnotesize $\backslash$#1}}
\def\env#1{\texttt{\color{blue}\footnotesize #1}}
\definecolor{deepblue}{rgb}{0,0,0.5}
\definecolor{deepred}{rgb}{0.6,0,0}
\definecolor{deepgreen}{rgb}{0,0.5,0}
\definecolor{halfgray}{gray}{0.55}

\lstset{
    basicstyle=\ttfamily\small,
    keywordstyle=\bfseries\color{deepblue},
    emphstyle=\ttfamily\color{deepred},    % Custom highlighting style
    stringstyle=\color{deepgreen},
    numbers=left,
    numberstyle=\small\color{halfgray},
    rulesepcolor=\color{red!20!green!20!blue!20},
    frame=shadowbox,
}


\begin{document}

\begin{frame}
    \titlepage
    \begin{figure}[htpb]
        \begin{center}
            \includegraphics[width=0.1\linewidth]{fib.png}
        \end{center}
    \end{figure}

    \begin{figure}[htpb]
        \begin{center}
            \includegraphics[width=0.1\linewidth]{Logo_UPC.png}
        \end{center}
    \end{figure}
\end{frame}

\begin{frame}
    \tableofcontents[sectionstyle=show,subsectionstyle=show/shaded/hide,subsubsectionstyle=show/shaded/hide]
\end{frame}


\section{Introducción}


\begin{frame}{?`Qué es el Software Libre?}

  \begin{itemize}
    \item \textbf{Las 4 libertades}
      \begin{itemize}
        \item Usar el programa
        \item Estudiar cómo funciona
        \item Modificarlo
        \item Compartirlo
      \end{itemize}

    \vspace{0.4cm}
    \item \textbf{Licencias habituales}: GPL, Apache, MIT

    \vspace{0.2cm}
    \item \textbf{Frente al software propietario}: código cerrado, sin derecho a modificar ni redistribuir
  \end{itemize}

\end{frame}


\begin{frame}{Software libre vs software propietario en móviles}

\begin{center}
\begin{tabular}{p{3cm} p{8cm}}
  \textcolor{red}{\textbf{Propietario}} & iOS, HarmonyOS (parcial), FireOS \\[6pt]
  \textcolor{orange}{\textbf{Mixto}}    & Android (AOSP libre + capa Google cerrada) \\[6pt]
  \textcolor{green!50!black}{\textbf{Libre}} & LineageOS, postmarketOS, /e/OS \\
\end{tabular}
\end{center}

\bigskip

\begin{itemize}
  \item Android es un punto intermedio: núcleo libre, servicios y capa de Google cerrados.
\end{itemize}

\end{frame}





\begin{frame}{Evolución del ecosistema móvil}

\begin{itemize}

  % --- TIMELINE ---
  \item \textbf{2000--2008}  
  \begin{itemize}
      \item Symbian, Windows Mobile
  \end{itemize}
  \vspace{6pt}

  \item \textbf{2010--hoy}  
  \begin{itemize}
      \item Dominio absoluto de iOS y Android
  \end{itemize}
  \vspace{6pt}

  \item \textbf{Actualidad}  
  \begin{itemize}
      \item Proyectos libres: LineageOS, postmarketOS, /e/OS
  \end{itemize}

\end{itemize}

\vspace{0.4cm}

\begin{itemize}
  \item Aunque el mercado está concentrado, las comunidades siguen impulsando alternativas libres.
\end{itemize}

\end{frame}



\begin{frame}{Tipos de dispositivos}

\begin{center}
    \includegraphics[width=1\linewidth]{dispositivos.png}
\end{center}

\end{frame}





\begin{frame}{Importancia del Software Libre en móviles}

\begin{columns}[T]  % Alineación superior

  \column{0.5\textwidth}
    \begin{block}{Privacidad}
      Mayor control sobre qué datos se recogen y cómo se usan.
    \end{block}

    \begin{block}{Transparencia}
      El código puede ser auditado por cualquiera.
    \end{block}

  \column{0.5\textwidth}
    \begin{block}{Control del dispositivo}
      Permite al usuario modificar y adaptar el sistema.
    \end{block}

    \begin{block}{Longevidad del hardware}
      Dispositivos sin soporte oficial pueden seguir actualizándose.
    \end{block}

\end{columns}

\vspace{0.4cm}

\begin{itemize}
  \item En móviles, tener opciones libres permite recuperar control y extender la vida útil del hardware.
\end{itemize}

\end{frame}

\begin{frame}{Limitaciones del ecosistema móvil actual}

\begin{itemize}
    \item \textbf{Bootloaders bloqueados}
    \item \textbf{Drivers propietarios}
    \item \textbf{Dependencia de tiendas de apps}
\end{itemize}

\vspace{0.4cm}

\begin{itemize}
    \item Las restricciones impuestas por los fabricantes limitan la instalación de sistemas libres, pero aún así existen proyectos que lo permiten.
\end{itemize}

\end{frame}

% Tema 2: Sistemas Operativos Libres

\section{Sistemas Operativos Libres para Dispositivos Móviles}

\begin{frame}
\frametitle{Situación actual de los SO para dispositivos móviles}

\begin{itemize}
\item \textbf{Oligopolio formado por:} \vspace{10pt}
    \begin{itemize}
        \item Google Android \vspace{2pt}
        \item Apple con iOS \vspace{2pt}
    \end{itemize}
\end{itemize}
\begin{center}
    \includegraphics[width=0.8\linewidth]{chart-mobileos.png}
\end{center}
\end{frame}

\begin{frame}
\frametitle{iOS}

\begin{itemize}
	\item Sistema operativo \textbf{exclusivo para smartphones de Apple} \vspace{10pt}
	\item Se caracteriza por su interfaz: \vspace{10pt}
	\begin{itemize}
		\item Simple y sencilla \vspace{2pt}
		\item Estilizada y moderna \vspace{2pt}
	\end{itemize}
\item Y también por su \textbf{ecosistema cerrado:}
	\begin{itemize}
		\item Personalización limitada\vspace{2pt}
		\item Compatibilidad con dispositivos de otras marcas limitada \vspace{2pt}
		\item \textbf{Monopolio de la distribución de aplicaciones}, procurándole demandas por monopolio a la compañia en territorio europeo
	\end{itemize}
\end{itemize}
\end{frame}

\begin{frame}
\frametitle{iOS}
    \begin{columns}[c] % [c] aligns them vertically in the center
    
        % Top Left Image
        \begin{column}{0.45\textwidth}
            \centering
            \includegraphics[width=\linewidth]{ios26-ui.jpeg}
        \end{column}

        % Top Right Image
        \begin{column}{0.45\textwidth}
            \centering
            \includegraphics[width=0.5\linewidth]{ios26-panel.jpeg}
        \end{column}
        
    \end{columns}

    \vspace{0.5cm} % Adjust this space between top and bottom

    \begin{center}
        \includegraphics[width=0.5\textwidth]{apple-lawsuit.png} 
    \end{center}
\end{frame}




\section[Negocio y seguridad]{Modelos de Negocio y Seguridad}

\begin{frame}
\frametitle{Modelos de Negocio y Seguridad}
\begin{center}
    \Huge \textbf{Modelos de Negocio y Seguridad}
\end{center}
\begin{center}
    \large 1. ?`Negocio viable? \\
    2. ?`Seguridad y privacidad?
\end{center}
\end{frame}

\begin{frame}
\frametitle{?`Se puede ganar dinero con software libre?}

\begin{itemize}
    \item !`POR SUPUESTOOO!  \vspace{10pt}
    \item No se basa en vender licencias. \vspace{10pt}
    \item Modelos comunes:\vspace{2pt}
    \begin{itemize}
        \item Servicios profesionales y personalización.\vspace{2pt}
        \item Soporte técnico especializado.\vspace{2pt}
        \item Desarrollo a medida para fabricantes o empresas.\vspace{2pt}
        \item Donaciones y financiación comunitaria. \vspace{10pt}
    \end{itemize}
    \item Oportunidades en ROMs, auditorías de seguridad, apps libres con servicios premium.
\end{itemize}

\end{frame}

\begin{frame}
\frametitle{Modelo 1: Donaciones y mecenazgo}

\begin{itemize}
    \item Modelo muy común en software libre móvil. \vspace{3pt}
    \item App libre y gratuita, y los usuarios contribuyen \textbf{voluntariamente}. \vspace{3pt}
    \item Patreon, Ko-fi o GitHub Sponsors permiten ingresos estables (sin cerrar el código ni imponer restricciones). \vspace{3pt}
    \item Comunidades \textbf{activas} y \textbf{comprometidas}. \vspace{3pt}
    \item Mecenas reciben \textbf{ventajas suaves} (acceso a betas, participación en decisiones, sin funciones privativas. \vspace{3pt}
    \item \textbf{Objetivo}: sostener el desarrollo a largo plazo manteniendo la libertad del software. \vspace{3pt}
\end{itemize}
\end{frame}

\begin{frame}
\frametitle{Modelo 2: Freemium y servicios añadidos}
\begin{itemize}
    \item App gratuita con \textbf{funciones premium} opcionales. \vspace{8pt}
    \item Monetización mediante \textbf{suscripciones} o pago único. \vspace{8pt}
    \item El código libre sin restricciones ni compromisos. \vspace{8pt}
    \item \textbf{Apps ideales:} \vspace{8pt} 
    \begin{itemize}
        \item{sincronización en la nube} \vspace{8pt}
        \item{colaboración entre usuarios} \vspace{8pt}
        \item{copias de seguridad} \vspace{8pt}
        \item{extensiones avanzadas} \vspace{8pt}
    \end{itemize}
\end{itemize}
\end{frame}

\begin{frame}
\frametitle{Modelo 3: Publicidad ética}
\begin{itemize}
    \item Uso de \textbf{anuncios no invasivos}. \vspace{8pt}
    \item No interrumpen la experiencia. \vspace{8pt}
    \item Publicidad \textbf{sin rastreo ni perfiles}. \vspace{8pt}
    \item Compromiso la \textbf{transparencia} hacia el usuario. \vspace{8pt}
    \item \textbf{Ejemplos de publicidad ética:} \vspace{8pt}
    \begin{itemize}
        \item banners estáticos sin seguimiento \vspace{8pt}
        \item anuncios no personalizados/perfilados y descentralizados \vspace{8pt}
        \item publicidad desactivable mediante donaciones \vspace{8pt}
    \end{itemize}
\end{itemize}
\end{frame}

\begin{frame}
\frametitle{Comparación de modelos y rentabilidad}
\begin{itemize}
    \item \textbf{Publicidad ética} \vspace{2pt}
    \begin{itemize}
        \item Ingresos bajos pero constantes \vspace{1pt}
        \item No invasiva y sin rastreo ((apps con muchos usuarios)) \vspace{1pt}
    \end{itemize} \vspace{2pt}

    \item \textbf{Donaciones} \vspace{2pt}
    \begin{itemize}
        \item Sostenible con comunidad activa \vspace{1pt}
        \item Ingresos impredecibles, dependen del compromiso \vspace{1pt}
    \end{itemize} \vspace{2pt}

    \item \textbf{Freemium y servicios añadidos} \vspace{2pt}
    \begin{itemize}
        \item Ingresos estables y escalables \vspace{1pt}
        \item Pago por funciones avanzadas o servicios adicionales \vspace{1pt}
    \end{itemize} \vspace{2pt}

    \item \textbf{Desarrollo por encargo y soporte} \vspace{2pt}
    \begin{itemize}
        \item Mayor rentabilidad y sostenibilidad a largo plazo\vspace{1pt}
        \item Personalización profesional y soporte técnico \vspace{1pt}
    \end{itemize}
\end{itemize}
\end{frame}


\begin{frame}
\frametitle{Seguridad en móviles: ?`es el SL más seguro?}

\begin{itemize}
    \item \textbf{Auditoría abierta} \vspace{5pt}
    \begin{itemize}
        \item Código \textbf{accesible} para cualquiera \vspace{3pt}
        \item Permite detectar \textbf{vulnerabilidades} de manera transparente \vspace{3pt}
    \end{itemize} \vspace{5pt}

    \item \textbf{Correcciones más rápidas} \vspace{5pt}
    \begin{itemize}
        \item Los fallos pueden ser \textbf{solucionados} por la \textbf{comunidad} \vspace{3pt}
        \item \textbf{Reduce el tiempo} entre detección y solucionado \vspace{3pt}
    \end{itemize} \vspace{5pt}

    \item \textbf{Sin componentes ocultos} \vspace{5pt}
    \begin{itemize}
        \item No hay puertas traseras escondidas \vspace{3pt}
        \item Garantiza mayor \textbf{privacidad} y \textbf{confianza} en la app \vspace{3pt}
    \end{itemize}
\end{itemize}
\end{frame}


\begin{frame}
\frametitle{Principales amenazas en móviles}

\begin{itemize}
    \item \textbf{Malware y falsificación de apps} \vspace{5pt}
    \begin{itemize}
        \item Riesgo de aplicaciones modificadas o maliciosas \vspace{3pt}
        \item Permite auditar y verificar el código \vspace{3pt}
    \end{itemize} \vspace{5pt}

    \item \textbf{Permisos excesivos} \vspace{5pt}
    \begin{itemize}
        \item Muchas apps solicitan más permisos de los necesarios \vspace{3pt}
        \item Revisar el código libre permite entender qué permisos realmente se usan \vspace{3pt}
    \end{itemize} \vspace{5pt}

    \item \textbf{Espionaje y rastreo} \vspace{5pt}
    \begin{itemize}
        \item Seguimiento de la actividad del usuario y recopilación de datos \vspace{3pt}
        \item El software libre aumenta la transparencia y la confianza \vspace{3pt}
    \end{itemize}
\end{itemize}
\end{frame}


\begin{frame}
\frametitle{Privacidad: microG, /e/OS y alternativas libres}

\begin{itemize}
    \item \textbf{microG} \vspace{5pt}
    \begin{itemize}
        \item Sustituye los servicios de Google \vspace{3pt}
        \item Reduce el rastreo manteniendo compatibilidad con apps \vspace{3pt}
    \end{itemize} \vspace{5pt}

    \item \textbf{/e/OS} \vspace{5pt}
    \begin{itemize}
        \item Sistema operativo móvil completamente libre \vspace{3pt}
        \item Prioriza la privacidad del usuario y minimiza datos compartidos \vspace{3pt}
    \end{itemize} \vspace{5pt}

    \item \textbf{LineageOS sin Google} \vspace{5pt}
    \begin{itemize}
        \item ROM basada en Android sin servicios propietarios \vspace{3pt}
        \item Permite controlar qué información se comparte \vspace{3pt}
    \end{itemize}
\end{itemize}
\end{frame}


\begin{frame}
\frametitle{Conclusión}

\begin{itemize}
    \item \textbf{Negocio viable} \vspace{3pt}
    \begin{itemize}
        \item Se pueden generar \textbf{ingresos} \vspace{3pt}
        \item Donaciones, freemium, publicidad ética y desarrollo por encargo \vspace{3pt}
    \end{itemize} \vspace{5pt}

    \item \textbf{Más seguridad y transparencia} \vspace{3pt}
    \begin{itemize}
        \item Código auditable y sin componentes ocultos \vspace{3pt}
        \item Corrección rápida de vulnerabilidades y control sobre permisos \vspace{3pt}
    \end{itemize} \vspace{5pt}

    \item \textbf{Futuro: más dispositivos abiertos} \vspace{3pt}
    \begin{itemize}
        \item Aumento de sistemas móviles libres y alternativos \vspace{3pt}
        \item Preocupación por la privacidad --> SL
    \end{itemize}
\end{itemize}

\end{frame}


\end{document}

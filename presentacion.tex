\documentclass{beamer}
\usepackage[utf8]{inputenc}
\usepackage[T1]{fontenc}
\usepackage[spanish]{babel}
% other packages
\usepackage{latexsym,amsmath,xcolor,multicol,booktabs,calligra}
\usepackage{graphicx,pstricks,listings,stackengine}

\author{Berta Ferré, Hajweria Hussain, Marc Ribas, Fardin Arafat, Francisco Ruiz}
\title{Software Libre en Dispositivos Móviles}
\subtitle{Sistemas operativos, impacto y desarrollo móvil}
\institute{Software Libre y Desarollo Social}
\date{\today}
\usepackage{CNU}

% defs
\def\cmd#1{\texttt{\color{red}\footnotesize $\backslash$#1}}
\def\env#1{\texttt{\color{blue}\footnotesize #1}}
\definecolor{deepblue}{rgb}{0,0,0.5}
\definecolor{deepred}{rgb}{0.6,0,0}
\definecolor{deepgreen}{rgb}{0,0.5,0}
\definecolor{halfgray}{gray}{0.55}

\lstset{
    basicstyle=\ttfamily\small,
    keywordstyle=\bfseries\color{deepblue},
    emphstyle=\ttfamily\color{deepred},    % Custom highlighting style
    stringstyle=\color{deepgreen},
    numbers=left,
    numberstyle=\small\color{halfgray},
    rulesepcolor=\color{red!20!green!20!blue!20},
    frame=shadowbox,
}


\begin{document}

\begin{frame}
    \titlepage
    \begin{figure}[htpb]
        \begin{center}
            \includegraphics[width=0.1\linewidth]{fib.png}
        \end{center}
    \end{figure}

    \begin{figure}[htpb]
        \begin{center}
            \includegraphics[width=0.1\linewidth]{Logo_UPC.png}
        \end{center}
    \end{figure}
\end{frame}

\begin{frame}
    \tableofcontents[sectionstyle=show,subsectionstyle=show/shaded/hide,subsubsectionstyle=show/shaded/hide]
\end{frame}

%------------------------------APARTADO 1---------------------------------

\section{Introducción}

\begin{frame}{?`Qué es el Software Libre?}

  \begin{itemize}
    \item \textbf{Las 4 libertades}
      \begin{itemize}
        \item Usar el programa
        \item Estudiar cómo funciona
        \item Modificarlo
        \item Compartirlo
      \end{itemize}

    \vspace{0.4cm}
    \item \textbf{Licencias habituales}: GPL, Apache, MIT

    \vspace{0.2cm}
    \item \textbf{Frente al software propietario}: código cerrado, sin derecho a modificar ni redistribuir
  \end{itemize}

\end{frame}


\begin{frame}{Software libre vs software propietario en móviles}

\begin{center}
\begin{tabular}{p{3cm} p{8cm}}
  \textcolor{red}{\textbf{Propietario}} & iOS, HarmonyOS (parcial), FireOS \\[6pt]
  \textcolor{orange}{\textbf{Mixto}}    & Android (AOSP libre + capa Google cerrada) \\[6pt]
  \textcolor{green!50!black}{\textbf{Libre}} & LineageOS, postmarketOS, /e/OS \\
\end{tabular}
\end{center}

\bigskip

\begin{itemize}
  \item Android es un punto intermedio: núcleo libre, servicios y capa de Google cerrados.
\end{itemize}

\end{frame}





\begin{frame}{Evolución del ecosistema móvil}

\begin{itemize}

  \item \textbf{2000--2008}
  \begin{itemize}
      \item Symbian, Windows Mobile
  \end{itemize}
  \vspace{6pt}

  \item \textbf{2010--hoy}
  \begin{itemize}
      \item Dominio absoluto de iOS y Android
  \end{itemize}
  \vspace{6pt}

  \item \textbf{Actualidad}
  \begin{itemize}
      \item Proyectos libres: LineageOS, postmarketOS, /e/OS
  \end{itemize}

\end{itemize}

\vspace{0.4cm}

\begin{itemize}
  \item Aunque el mercado está concentrado, las comunidades siguen impulsando alternativas libres.
\end{itemize}

\end{frame}



\begin{frame}{Tipos de dispositivos}

\begin{center}
    \includegraphics[width=1\linewidth]{dispositivos.png}
\end{center}

\end{frame}





\begin{frame}{Importancia del Software Libre en móviles}

\begin{columns}[T]

  \column{0.5\textwidth}
    \begin{block}{Privacidad}
      Mayor control sobre qué datos se recogen y cómo se usan.
    \end{block}

    \begin{block}{Transparencia}
      El código puede ser auditado por cualquiera.
    \end{block}

  \column{0.5\textwidth}
    \begin{block}{Control del dispositivo}
      Permite al usuario modificar y adaptar el sistema.
    \end{block}

    \begin{block}{Longevidad del hardware}
      Dispositivos sin soporte oficial pueden seguir actualizándose.
    \end{block}

\end{columns}

\vspace{0.4cm}

\begin{itemize}
  \item En móviles, tener opciones libres permite recuperar control y extender la vida útil del hardware.
\end{itemize}

\end{frame}

\begin{frame}{Limitaciones del ecosistema móvil actual}

\begin{itemize}
    \item \textbf{Bootloaders bloqueados}
    \item \textbf{Drivers propietarios}
    \item \textbf{Dependencia de tiendas de apps}
\end{itemize}

\vspace{0.4cm}

\begin{itemize}
    \item Las restricciones impuestas por los fabricantes limitan la instalación de sistemas libres, pero aún así existen proyectos que lo permiten.
\end{itemize}

\end{frame}

%------------------------------APARTADO 2---------------------------------

\section[SO Libres]{SO Libres en Dispositivos Móviles}

\begin{frame}
\frametitle{Contexto}

\begin{itemize}
\item \textbf{Oligopolio formado por:} \vspace{10pt}
    \begin{itemize}
        \item Google Android \vspace{2pt}
        \item iOS \vspace{2pt}
    \end{itemize}
\end{itemize}
\begin{center}
    \includegraphics[width=0.8\linewidth]{chart-mobileos.png}
\end{center}
\end{frame}

\begin{frame}
\frametitle{iOS}

\begin{itemize}
	\item Sistema operativo \textbf{exclusivo para smartphones de Apple} \vspace{10pt}
	\item Se caracteriza por su interfaz: \vspace{10pt}
\item Y también por su \textbf{ecosistema cerrado:}
	\begin{itemize}
		\item Personalización limitada\vspace{2pt}
		\item Compatibilidad con dispositivos de otras marcas limitada \vspace{2pt}
		\item \textbf{Monopolio de la distribución de aplicaciones}, procurándole demandas a Apple por en territorio europeo
	\end{itemize}
\end{itemize}
\end{frame}

\begin{frame}
\frametitle{Google Android}

\begin{itemize}
	\item Sistema operativo creado a partir del \textbf{Android Open Source Project (AOSP)}, añadiendo una capa de software propietario: los \textbf{Google Mobile Services} \vspace{10pt}
	\item Funciones implementadas con los Google Mobile Services: \vspace{10pt}
	\begin{itemize}
		\item Gestión unificada de las notificaciones \vspace{2pt}
		\item QuickShare \vspace{2pt}
		\item Geolocalización precisa \vspace{2pt}
		\item Google Wallet\vspace{2pt}
		\item ...\vspace{2pt}
	\end{itemize}
\end{itemize}
\end{frame}


\begin{frame}
\frametitle{Google Android}

\begin{itemize}
	\item \textbf{Problemas:}\vspace{10pt}
	\begin{itemize}
		\item Muchas aplicaciones de Android dependen de los Google Mobile Services \vspace{2pt}
		\item Google tiene acceso constante a tu dispositivo, tus datos y tu actividad\vspace{2pt}
		\item A partir de 2026, Google usará la \textbf{Play Integrity API} para bloquear aplicaciones que no esten verificadas por la Play Store\vspace{2pt}
	\end{itemize}
\end{itemize}
\end{frame}

\begin{frame}
\frametitle{Opciones de Sistemas Operativos Libres}

\begin{itemize}
	\item \textbf{Forks del AOSP:} GrapheneOS, LineageOS \vspace{10pt}
	\item \textbf{Linux:} UbuntuTouch, PostmarketOS \vspace{10pt}
\end{itemize}
\begin{columns}[c] % [c] aligns them vertically in the center

% Top Left Image
\begin{column}{0.45\textwidth}
    \centering
	\includegraphics[width=1.2\linewidth]{AOSP.png}
\end{column}

% Top Right Image
\begin{column}{0.45\textwidth}
    \centering
    \includegraphics[width=0.8\linewidth]{linux-mobile.png}
\end{column}

\end{columns}
\end{frame}

\begin{frame}
\frametitle{GrapheneOS}

\begin{columns}[c]
\begin{column}{0.8\textwidth}
	\begin{itemize}
		\item Fork del AOSP orientado a: \textbf{Privacidad} y \textbf{Seguridad} \vspace{10pt}
		\item No incluye los Google Mobile Services \vspace{10pt}
		\item Control granular de todos los permisos de las aplicaciones \vspace{10pt}
		\item Mejoras en Android para aumentar la seguridad: \vspace{10pt}
			\begin{itemize}
			\item Sandboxes
			\item Randomizado de MAC
			\item Sistemas de auditoria
			\item Perfiles aislados
			\item Ejecuciones seguras
			\item ...
			\end{itemize}
	\end{itemize}
	\end{column}
\begin{column}{0.2\textwidth}
            \includegraphics[width=1\linewidth]{grapheneos.png}
\end{column}

    \end{columns}
\end{frame}

\begin{frame}
\frametitle{GrapheneOS}

\begin{itemize}
	\item \textbf{Inconvenientes:} \vspace{10pt}
	\begin{itemize}
		\item Requiere un \textbf{módulo TPM}, únicamente Google Pixel \vspace{2pt}
		\item Apps que requieran Google Mobile Services \vspace{2pt}
	\end{itemize}
	\item \textbf{Solución a los Google Mobile Services:} Sandboxed Google Services
\end{itemize}
\end{frame}

\begin{frame}
\frametitle{Linux en Dispositivos Móviles}

\begin{itemize}
	\item Kernel de Linux compatible con CPUs ARM \vspace{10pt}
	\item Adapta software de Desktop a Mobile como Wayland, GNOME, KDE, etc.\vspace{10pt}
	\item \textbf{Waydroid:} Capa de compatibilidad con aplicaciones Android\vspace{10pt}
	\item El problema son los \textbf{drivers propietarios} de antenas, cámaras, módems, etc.\vspace{10pt}
\end{itemize}
\end{frame}

\begin{frame}
\frametitle{Linux en Dispositivos Móviles}

\begin{itemize}
	\item \textbf{UbuntuTouch:} \vspace{10pt}
	\end{itemize}
	\centering
	\includegraphics[width=0.4\linewidth]{ubuntu-touch.jpg}
	\begin{itemize}
	\item \textbf{PostmarketOS:} \vspace{10pt}
	\end{itemize}
	\centering
	\includegraphics[width=0.4\linewidth]{postmarketos.jpg}
\end{frame}

\begin{frame}
\frametitle{Conclusiones}

\begin{itemize}
	\item Google Android y iOS son muy convenientes y populares, pero \textbf{comprometen nuestra libertad y privacidad} \vspace{10pt}
	\item Ya existen alternativas muy usables y buenas a partir de forks del AOSP como \textbf{GrapheneOS o LineageOS} \vspace{10pt}
	\item Linux en dispositivos móviles aún necesita desarrollo \vspace{10pt}
\end{itemize}
\end{frame}

%------------------------------APARTADO 3---------------------------------


\section[Aplicaciones de SL para Móviles]{Aplicaciones de Software Libre para Móviles}


\begin{frame}
\frametitle{Ecosistema de Apps Libres}
\begin{itemize}
    \item \textbf{F-Droid}
        \begin{itemize}
            \item Portal de aplicaciones basado en software libre.
            \item Paquetes compilados directamente desde su código fuente.
            \item No hace falta iniciar sesión para descargar o instalar.
        \end{itemize}
\end{itemize}
\end{frame}

\begin{frame}
\frametitle{Categorías Principales de Apps Libres}
\begin{itemize}
    \item Comunicación: Signal, Element, Briar
    \pause
    \item Navegación web: Firefox, Brave, Tor Browser
    \pause
    \item Navegación y Mapas: OsmAnd, OpenStreetMap
    \pause
    \item Multimedia: VLC, Fossify Gallery
\end{itemize}
\end{frame}

\begin{frame}
\frametitle{Comparativa con Apps Propietarias}
\begin{center}
\small
\begin{tabular}{|p{0.3\textwidth}|p{0.3\textwidth}|p{0.3\textwidth}|}
\hline
\textbf{Aspecto} & \textbf{Software Libre} & \textbf{Software Propietario} \\
\hline
Privacidad & Alta  & Variable \\
\hline
Seguridad & Código auditable por la comunidad & Cerrado\\
\hline
Coste & Gratuito & Gratuito o de pago \\
\hline
Publicidad & Ninguna & Frecuente en versión gratuita \\
\hline
Personalización & Total & Limitada \\
\hline
\end{tabular}
\end{center}
\end{frame}

\begin{frame}
\frametitle{iOS y Software Libre}
\begin{itemize}
    \item Sistema cerrado
    \item Instalación de apps solo desde App Store
    \item Pocas apps libres disponibles debido a las restricciones de la plataforma.
    \item Digital Markets Act y posibles cambios futuros.
\end{itemize}
\end{frame}

\begin{frame}
\frametitle{Conclusión}
\begin{itemize}
    \item Es VIABLE
    \item Independencia de grandes corporaciones.
    \item Acceso a código fuente y auditoría.
    \item Fomentar el uso de apps libres contribuye a la privacidad y soberanía digital.
    \item Recuperar el control y el derecho a la privacidad.
\end{itemize}
\end{frame}

%------------------------------APARTADO 4---------------------------------
\section[Apps libres]{Software Libre en Aplicaciones Móviles: Desarrollo, Costes y Difusión}

%-----------

\begin{frame}
\frametitle{Ecosistema de Apps Libres}

\begin{center}
\large \textbf{¿Dónde se encuentran las apps libres?}
\end{center}

\vspace{0.3cm}

\begin{itemize}
  \item Repositorios comunitarios
        \vspace{10pt}

  \item Alternativas libres a tiendas privativas
        \vspace{10pt}

  \item \textbf{Beneficios clave:}
  \vspace{2pt}
  \begin{itemize}
    \item Control total
    \vspace{5pt}
    \item Protección de la privacidad
    \vspace{5pt}
    \item Transparencia del código
  \end{itemize}
\end{itemize}

\end{frame}

%----------

\begin{frame}
\frametitle{¿Cómo se desarrolla una app libre?}
\begin{itemize}
  \item \textbf{Android} → Kotlin / Java
        \vspace{10pt}

  \item \textbf{iOS} → Swift (con restricciones impuestas)
        \vspace{10pt}

  \item \textbf{Frameworks libres y multiplataforma:}
  \begin{itemize}
      \item Flutter
      \item React Native
      \item Qt
      \item Kivy
  \end{itemize}
  \vspace{10pt}

  \item \textbf{Ventajas del desarrollo libre:}
  \begin{itemize}
    \item Código auditable
    \item Control total del proyecto
    \item Colaboración comunitaria
  \end{itemize}
\end{itemize}
\end{frame}

%------------

\begin{frame}
\frametitle{Costes de Publicación}

\begin{center}
\large \textbf{Publicar una app… ¿cuánto cuesta realmente?}
\end{center}

\vspace{0.4cm}

\begin{itemize}
  \item \textbf{Google Play:} Tarifa y comisión por ventas
  \vspace{5pt}
  \item \textbf{Apple App Store:} Pago anual obligatorio
  \vspace{5pt}
  \item \textbf{F-Droid:} Publicación completamente gratuita
\end{itemize}

\vspace{0.5cm}

\begin{center}
\textbf{Ventaja del software libre:} Sin barreras económicas.
\end{center}

\end{frame}

%-----------

\begin{frame}
\frametitle{Cómo Difundir Software Libre}

\begin{center}
   \textbf{Si no hay publicidad… ¿cómo se dan a conocer?}
\end{center}

\vspace{0.5cm}

\begin{itemize}
  \item \textbf{Comunidad:} Foros, GitHub, Mastodon
  \vspace{10pt}
  \item Documentación y tutoriales accesibles
  \vspace{10pt}
  \item Marketing limitado pero auténtico
  \vspace{10pt}
  \item Crecimiento impulsado por privacidad y ética
\end{itemize}

\end{frame}

%----------

\begin{frame}
\frametitle{Conclusión}

\begin{center}
\large \textbf{El software libre en móviles… ¿es el futuro?}
\end{center}

\vspace{0.5cm}

\begin{itemize}
  \item Desarrollo libre = Control + Transparencia
  \vspace{5pt}
  \item Publicación gratuita en F-Droid
  \vspace{5pt}
  \item Difusión basada en comunidad
  \vspace{5pt}
  \item Ecosistema viable y sostenible
\end{itemize}

\vspace{0.5cm}

\begin{center}
\large \textit{¿Qué pasaría si todo el software móvil fuera libre?}
\end{center}

\end{frame}

%------------------------------APARTADO 5---------------------------------
\section[Negocio y seguridad]{Modelos de Negocio y Seguridad}


\begin{frame}
\frametitle{?`Se puede ganar dinero con software libre?}

\begin{itemize}
    \item !`POR SUPUESTOOO!  \vspace{10pt}
    \item No se basa en vender licencias. \vspace{10pt}
    \item Modelos comunes:\vspace{2pt}
    \begin{itemize}
        \item Servicios profesionales y personalización.\vspace{2pt}
        \item Soporte técnico especializado.\vspace{2pt}
        \item Desarrollo a medida para fabricantes o empresas.\vspace{2pt}
        \item Donaciones y financiación comunitaria. \vspace{10pt}
    \end{itemize}
    \item Oportunidades en ROMs, auditorías de seguridad, apps libres con servicios premium.
\end{itemize}

\end{frame}

\begin{frame}
\frametitle{Modelo 1: Donaciones y mecenazgo}

\begin{itemize}
    \item Modelo muy común en software libre móvil. \vspace{3pt}
    \item App libre y gratuita, y los usuarios contribuyen \textbf{voluntariamente}. \vspace{3pt}
    \item Patreon, Ko-fi o GitHub Sponsors permiten ingresos estables (sin cerrar el código ni imponer restricciones). \vspace{3pt}
    \item Comunidades \textbf{activas} y \textbf{comprometidas}. \vspace{3pt}
    \item Mecenas reciben \textbf{ventajas suaves} (acceso a betas, participación en decisiones, sin funciones privativas. \vspace{3pt}
    \item \textbf{Objetivo}: sostener el desarrollo a largo plazo manteniendo la libertad del software. \vspace{3pt}
\end{itemize}
\end{frame}

\begin{frame}
\frametitle{Modelo 2: Freemium y servicios añadidos}
\begin{itemize}
    \item App gratuita con \textbf{funciones premium} opcionales. \vspace{8pt}
    \item Monetización mediante \textbf{suscripciones} o pago único. \vspace{8pt}
    \item El código libre sin restricciones ni compromisos. \vspace{8pt}
    \item \textbf{Apps ideales:} \vspace{8pt}
    \begin{itemize}
        \item{sincronización en la nube} \vspace{8pt}
        \item{colaboración entre usuarios} \vspace{8pt}
        \item{copias de seguridad} \vspace{8pt}
        \item{extensiones avanzadas} \vspace{8pt}
    \end{itemize}
\end{itemize}
\end{frame}

\begin{frame}
\frametitle{Modelo 3: Publicidad ética}
\begin{itemize}
    \item Uso de \textbf{anuncios no invasivos}. \vspace{8pt}
    \item No interrumpen la experiencia. \vspace{8pt}
    \item Publicidad \textbf{sin rastreo ni perfiles}. \vspace{8pt}
    \item Compromiso la \textbf{transparencia} hacia el usuario. \vspace{8pt}
    \item \textbf{Ejemplos de publicidad ética:} \vspace{8pt}
    \begin{itemize}
        \item banners estáticos sin seguimiento \vspace{8pt}
        \item anuncios no personalizados/perfilados y descentralizados \vspace{8pt}
        \item publicidad desactivable mediante donaciones \vspace{8pt}
    \end{itemize}
\end{itemize}
\end{frame}

\begin{frame}
\frametitle{Comparación de modelos y rentabilidad}
\begin{itemize}
    \item \textbf{Publicidad ética} \vspace{2pt}
    \begin{itemize}
        \item Ingresos bajos pero constantes \vspace{1pt}
        \item No invasiva y sin rastreo ((apps con muchos usuarios)) \vspace{1pt}
    \end{itemize} \vspace{2pt}

    \item \textbf{Donaciones} \vspace{2pt}
    \begin{itemize}
        \item Sostenible con comunidad activa \vspace{1pt}
        \item Ingresos impredecibles, dependen del compromiso \vspace{1pt}
    \end{itemize} \vspace{2pt}

    \item \textbf{Freemium y servicios añadidos} \vspace{2pt}
    \begin{itemize}
        \item Ingresos estables y escalables \vspace{1pt}
        \item Pago por funciones avanzadas o servicios adicionales \vspace{1pt}
    \end{itemize} \vspace{2pt}

    \item \textbf{Desarrollo por encargo y soporte} \vspace{2pt}
    \begin{itemize}
        \item Mayor rentabilidad y sostenibilidad a largo plazo\vspace{1pt}
        \item Personalización profesional y soporte técnico \vspace{1pt}
    \end{itemize}
\end{itemize}
\end{frame}


\begin{frame}
\frametitle{Seguridad en móviles: ?`es el SL más seguro?}

\begin{itemize}
    \item \textbf{Auditoría abierta} \vspace{5pt}
    \begin{itemize}
        \item Código \textbf{accesible} para cualquiera \vspace{3pt}
        \item Permite detectar \textbf{vulnerabilidades} de manera transparente \vspace{3pt}
    \end{itemize} \vspace{5pt}

    \item \textbf{Correcciones más rápidas} \vspace{5pt}
    \begin{itemize}
        \item Los fallos pueden ser \textbf{solucionados} por la \textbf{comunidad} \vspace{3pt}
        \item \textbf{Reduce el tiempo} entre detección y solucionado \vspace{3pt}
    \end{itemize} \vspace{5pt}

    \item \textbf{Sin componentes ocultos} \vspace{5pt}
    \begin{itemize}
        \item No hay puertas traseras escondidas \vspace{3pt}
        \item Garantiza mayor \textbf{privacidad} y \textbf{confianza} en la app \vspace{3pt}
    \end{itemize}
\end{itemize}
\end{frame}


\begin{frame}
\frametitle{Principales amenazas en móviles}

\begin{itemize}
    \item \textbf{Malware y falsificación de apps} \vspace{5pt}
    \begin{itemize}
        \item Riesgo de aplicaciones modificadas o maliciosas \vspace{3pt}
        \item Permite auditar y verificar el código \vspace{3pt}
    \end{itemize} \vspace{5pt}

    \item \textbf{Permisos excesivos} \vspace{5pt}
    \begin{itemize}
        \item Muchas apps solicitan más permisos de los necesarios \vspace{3pt}
        \item Revisar el código libre permite entender qué permisos realmente se usan \vspace{3pt}
    \end{itemize} \vspace{5pt}

    \item \textbf{Espionaje y rastreo} \vspace{5pt}
    \begin{itemize}
        \item Seguimiento de la actividad del usuario y recopilación de datos \vspace{3pt}
        \item El software libre aumenta la transparencia y la confianza \vspace{3pt}
    \end{itemize}
\end{itemize}
\end{frame}


\begin{frame}
\frametitle{Privacidad: microG, /e/OS y alternativas libres}

\begin{itemize}
    \item \textbf{microG} \vspace{5pt}
    \begin{itemize}
        \item Sustituye los servicios de Google \vspace{3pt}
        \item Reduce el rastreo manteniendo compatibilidad con apps \vspace{3pt}
    \end{itemize} \vspace{5pt}

    \item \textbf{/e/OS} \vspace{5pt}
    \begin{itemize}
        \item Sistema operativo móvil completamente libre \vspace{3pt}
        \item Prioriza la privacidad del usuario y minimiza datos compartidos \vspace{3pt}
    \end{itemize} \vspace{5pt}

    \item \textbf{LineageOS sin Google} \vspace{5pt}
    \begin{itemize}
        \item ROM basada en Android sin servicios propietarios \vspace{3pt}
        \item Permite controlar qué información se comparte \vspace{3pt}
    \end{itemize}
\end{itemize}
\end{frame}


\begin{frame}
\frametitle{Conclusión}

\begin{itemize}
    \item \textbf{Negocio viable} \vspace{3pt}
    \begin{itemize}
        \item Se pueden generar \textbf{ingresos} \vspace{3pt}
        \item Donaciones, freemium, publicidad ética y desarrollo por encargo \vspace{3pt}
    \end{itemize} \vspace{5pt}

    \item \textbf{Más seguridad y transparencia} \vspace{3pt}
    \begin{itemize}
        \item Código auditable y sin componentes ocultos \vspace{3pt}
        \item Corrección rápida de vulnerabilidades y control sobre permisos \vspace{3pt}
    \end{itemize} \vspace{5pt}

    \item \textbf{Futuro: más dispositivos abiertos} \vspace{3pt}
    \begin{itemize}
        \item Aumento de sistemas móviles libres y alternativos \vspace{3pt}
        \item Preocupación por la privacidad --> SL
    \end{itemize}
\end{itemize}

\end{frame}


%------------------------ CONCLUSIÓN FINAL ------------------------
\section{Conclusión}

\begin{frame}{Conclusiones y reflexiones}
\begin{itemize}
    \item La mayoría del software móvil es cerrado, poco control del usuario.
    \item El software libre aporta transparencia y autonomía.
    \item Existen alternativas libres que crecen por la demanda de más privacidad.
    \item Cambiar el SO es complicado, pero posible, y devuelve control al dispositivo.
\end{itemize}
\end{frame}

\begin{frame}
    \titlepage
    \begin{figure}[htpb]
        \begin{center}
            \includegraphics[width=0.1\linewidth]{fib.png}
        \end{center}
    \end{figure}

    \begin{figure}[htpb]
        \begin{center}
            \includegraphics[width=0.1\linewidth]{Logo_UPC.png}
        \end{center}
    \end{figure}
\end{frame}

\end{document}

\documentclass{beamer}
\usetheme{UASLP1}
\usepackage[utf8]{inputenc}
\usepackage[T1]{fontenc}

%% Use any fonts you like.
\usepackage{helvet}

\title{Software Libre en Dispositivos Móviles}
\subtitle{Explorando el impacto, los sistemas operativos y el desarrollo en plataformas móviles}
\author{Berta Ferré, Hajweria, Marc, Fardin, Fran Ruiz}
\date{\today}
\institute{Software libre y desarrollo social}



\begin{document}

\begin{frame}[plain,t]
\titlepage
\end{frame}

\section{Introducción}

\begin{frame}
\frametitle{Introducción}
\begin{center}
    \Huge \textbf{Introducción al Software Libre en Dispositivos Móviles}
\end{center}
\begin{center}
    \large Una visión general sobre el impacto del software libre en el ámbito móvil.
\end{center}
\end{frame}

\begin{frame}
\frametitle{Objetivos}
\begin{itemize}
  \item \textbf{Objetivo:} Analizar el papel del software libre en móviles, tablets, UMPCs y MIDs.
  \item Presentación del grupo.
  \item Relevancia social del software libre en la movilidad.
\end{itemize}
\end{frame}

\begin{frame}
\frametitle{¿Qué es el Software Libre?}
\begin{itemize}
  \item Las 4 libertades: usar, estudiar, compartir, y modificar el software.
  \item Ejemplos de licencias: GPL, Apache, MIT.
  \item Diferencias con "código abierto" y software propietario.
\end{itemize}
\end{frame}

\begin{frame}
\frametitle{Historia del Software Libre en Móviles}
\begin{itemize}
  \item OpenMoko, Maemo, MeeGo.
  \item Nacimiento y expansión de Android.
  \item Influencia del software libre en la evolución del entorno móvil.
\end{itemize}
\end{frame}

\begin{frame}
\frametitle{Impacto Social del Software Libre Móvil}
\begin{itemize}
  \item Democratización del acceso.
  \item Soberanía digital.
  \item Independencia tecnológica y transparencia.
  \item Beneficios sociales y educativos.
\end{itemize}
\end{frame}

\section{SO Móviles: Cerrados vs Abiertos}

\begin{frame}
\frametitle{SO Móviles: Cerrados vs Abiertos}
\begin{center}
    \Huge \textbf{SO Móviles: Cerrados vs Abiertos}
\end{center}
\begin{center}
    \large Un análisis sobre las diferencias entre los sistemas operativos cerrados y abiertos.
\end{center}
\end{frame}

\begin{frame}
\frametitle{Clasificación de SO Móviles}
\begin{itemize}
  \item Sistemas operativos cerrados, abiertos y libres.
  \item Definir los tres modelos y su relación con el control del usuario.
\end{itemize}
\end{frame}

\begin{frame}
\frametitle{Comparación Android, iOS, KaiOS, LineageOS}
\begin{itemize}
  \item Comparación de apertura, privacidad y ecosistema.
  \item Android no es 100\% libre.
  \item El monopolio del App Store en iOS.
\end{itemize}
\end{frame}

\begin{frame}
\frametitle{¿Qué significa que un SO sea libre?}
\begin{itemize}
  \item Importancia de drivers libres y bootloaders abiertos.
  \item Problemas actuales: blobs propietarios y firmware cerrado.
\end{itemize}
\end{frame}

\begin{frame}
\frametitle{Impacto Ético y Social de los Sistemas Cerrados}
\begin{itemize}
  \item Dependencia corporativa.
  \item Riesgos para la privacidad.
  \item Relación entre software cerrado y control del usuario.
\end{itemize}
\end{frame}

\section{SO Libres: ¿Se Puede Cambiar el SO?}

\begin{frame}
\frametitle{SO Libres: ¿Se Puede Cambiar el SO?}
\begin{center}
    \Huge \textbf{SO Libres: ¿Se Puede Cambiar el SO?}
\end{center}
\begin{center}
    \large Explorando la viabilidad de cambiar el sistema operativo en dispositivos móviles.
\end{center}
\end{frame}

\begin{frame}
\frametitle{Sistemas Realmente Libres}
\begin{itemize}
  \item Ubuntu Touch, postmarketOS, Tizen, KDE Plasma Mobile.
  \item Qué ofrece cada sistema y su comunidad.
\end{itemize}
\end{frame}

\begin{frame}
\frametitle{¿Es Posible Cambiar el SO del Dispositivo?}
\begin{itemize}
  \item Bootloaders, métodos de flasheo, bloqueos.
  \item Qué limita actualmente la instalación de sistemas alternativos.
\end{itemize}
\end{frame}

\begin{frame}
\frametitle{Ventajas y Riesgos del Cambio}
\begin{itemize}
  \item Privacidad, soporte a largo plazo vs garantía, estabilidad.
  \item Por qué no es una práctica masiva.
\end{itemize}
\end{frame}

\begin{frame}
\frametitle{Ejemplos Reales}
\begin{itemize}
  \item PinePhone, Librem 5, GPD.
  \item Estado actual y retos.
\end{itemize}
\end{frame}

\section{Software Libre en Aplicaciones Móviles: Desarrollo, Costes y Difusión}

\begin{frame}
\frametitle{Software Libre en Aplicaciones Móviles: Desarrollo, Costes y Difusión}
\begin{center}
    \Huge \textbf{Software Libre en Aplicaciones Móviles: Desarrollo, Costes y Difusión}
\end{center}
\begin{center}
    \large Explorando el desarrollo, los costes y la difusión del software libre en el ámbito móvil.
\end{center}
\end{frame}

\begin{frame}
\frametitle{Ecosistema de Apps Libres}
\begin{itemize}
  \item F-Droid, repositorios comunitarios, alternativas libres.
  \item Por qué F-Droid es clave para el software libre en móviles.
\end{itemize}
\end{frame}

\begin{frame}
\frametitle{Cómo Desarrollar Apps Libres}
\begin{itemize}
  \item Android (Kotlin/Java), iOS (Swift pero con restricciones), frameworks libres.
  \item Ventajas e inconvenientes de cada entorno.
\end{itemize}
\end{frame}

\begin{frame}
\frametitle{Costes de Publicación}
\begin{itemize}
  \item Coste de publicación en Google Play.
  \item Tarifas Apple.
  \item Publicación gratuita en F-Droid.
  \item Comparativa clara y directa.
\end{itemize}
\end{frame}

\begin{frame}
\frametitle{Cómo Difundir Software Libre}
\begin{itemize}
  \item Comunidad, documentación, visibilidad.
  \item Importancia del soporte comunitario y transparencia.
\end{itemize}
\end{frame}


% ------------------------------------------------------------------


\section{Modelos de Negocio y Seguridad}

\begin{frame}
\frametitle{Modelos de Negocio y Seguridad}
\begin{center}
    \Huge \textbf{Modelos de Negocio y Seguridad}
\end{center}
\begin{center}
    \large 1. ¿Negocio viable? \\
    2. ¿Seguridad y privacidad?
\end{center}
\end{frame}

\begin{frame}
\frametitle{¿Se puede ganar dinero con software libre?}

\begin{itemize}
    \item ¿POR SUPUESTOOO!  \vspace{10pt}
    \item No se basa en vender licencias. \vspace{10pt}
    \item Modelos comunes:\vspace{2pt}
    \begin{itemize}
        \item Servicios profesionales y personalización.\vspace{2pt}
        \item Soporte técnico especializado.\vspace{2pt}
        \item Desarrollo a medida para fabricantes o empresas.\vspace{2pt}
        \item Donaciones y financiación comunitaria. \vspace{10pt}
    \end{itemize}
    \item Oportunidades en ROMs, auditorías de seguridad, apps libres con servicios premium.
\end{itemize}

\end{frame}

\begin{frame}
\frametitle{Modelo 1: Donaciones y mecenazgo}

\begin{itemize}
    \item Modelo muy común en software libre móvil. \vspace{3pt}
    \item App libre y gratuita, y los usuarios contribuyen \textbf{voluntariamente}. \vspace{3pt}
    \item Patreon, Ko-fi o GitHub Sponsors permiten ingresos estables (sin cerrar el código ni imponer restricciones). \vspace{3pt}
    \item Comunidades \textbf{activas} y \textbf{comprometidas}. \vspace{3pt}
    \item Mecenas reciben \textbf{ventajas suaves} (acceso a betas, participación en decisiones, sin funciones privativas. \vspace{3pt}
    \item \textbf{Objetivo}: sostener el desarrollo a largo plazo manteniendo la libertad del software. \vspace{3pt}
\end{itemize}
\end{frame}

\begin{frame}
\frametitle{Modelo 2: Freemium y servicios añadidos}
\begin{itemize}
    \item App gratuita con \textbf{funciones premium} opcionales. \vspace{8pt}
    \item Monetización mediante \textbf{suscripciones} o pago único. \vspace{8pt}
    \item El código libre sin restricciones ni compromisos. \vspace{8pt}
    \item \textbf{Apps ideales:} \vspace{8pt} 
    \begin{itemize}
        \item{sincronización en la nube} \vspace{8pt}
        \item{colaboración entre usuarios} \vspace{8pt}
        \item{copias de seguridad} \vspace{8pt}
        \item{extensiones avanzadas} \vspace{8pt}
    \end{itemize}
\end{itemize}
\end{frame}

\begin{frame}
\frametitle{Modelo 3: Publicidad ética}
\begin{itemize}
    \item Uso de \textbf{anuncios no invasivos}. \vspace{8pt}
    \item No interrumpen la experiencia. \vspace{8pt}
    \item Publicidad \textbf{sin rastreo ni perfiles}. \vspace{8pt}
    \item Compromiso la \textbf{transparencia} hacia el usuario. \vspace{8pt}
    \item \textbf{Ejemplos de publicidad ética:} \vspace{8pt}
    \begin{itemize}
        \item banners estáticos sin seguimiento \vspace{8pt}
        \item anuncios no personalizados/perfilados y descentralizados \vspace{8pt}
        \item publicidad desactivable mediante donaciones \vspace{8pt}
    \end{itemize}
\end{itemize}
\end{frame}

\begin{frame}
\frametitle{Comparación de modelos y rentabilidad}
\begin{itemize}
    \item \textbf{Publicidad ética} \vspace{2pt}
    \begin{itemize}
        \item Ingresos bajos pero constantes \vspace{1pt}
        \item No invasiva y sin rastreo ((apps con muchos usuarios)) \vspace{1pt}
    \end{itemize} \vspace{2pt}

    \item \textbf{Donaciones} \vspace{2pt}
    \begin{itemize}
        \item Sostenible con comunidad activa \vspace{1pt}
        \item Ingresos impredecibles, dependen del compromiso \vspace{1pt}
    \end{itemize} \vspace{2pt}

    \item \textbf{Freemium y servicios añadidos} \vspace{2pt}
    \begin{itemize}
        \item Ingresos estables y escalables \vspace{1pt}
        \item Pago por funciones avanzadas o servicios adicionales \vspace{1pt}
    \end{itemize} \vspace{2pt}

    \item \textbf{Desarrollo por encargo y soporte} \vspace{2pt}
    \begin{itemize}
        \item Mayor rentabilidad y sostenibilidad a largo plazo\vspace{1pt}
        \item Personalización profesional y soporte técnico \vspace{1pt}
    \end{itemize}
\end{itemize}
\end{frame}





\begin{frame}
\frametitle{Seguridad y Software Libre}
\begin{itemize}
  \item Auditoría de código.
  \item Criptografía y E2EE.
  \item Por qué el software libre favorece la transparencia.
\end{itemize}
\end{frame}

\begin{frame}
\frametitle{Soberanía Digital y Futuro}
\begin{itemize}
  \item Independencia de proveedores.
  \item Tendencias futuras.
  \item Importancia estratégica del software libre.
\end{itemize}
\end{frame}

\end{document}

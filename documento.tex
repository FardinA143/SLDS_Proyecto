\documentclass[a4paper,12pt]{report}
\usepackage[utf8]{inputenc}
\usepackage[spanish]{babel}
\usepackage{graphicx}
\usepackage[colorlinks=true, linkcolor=black, urlcolor=black, citecolor=black]{hyperref}
\usepackage{lipsum}
\usepackage{titlesec}
\usepackage{tocloft}
\usepackage{geometry}

\linespread{1.3}

\geometry{left=2.5cm, right=2.5cm, top=3cm, bottom=3cm}

\renewcommand{\cftsecfont}{\normalfont}
\renewcommand{\cftsubsecfont}{\normalfont}
\renewcommand{\cftsecpagefont}{\normalfont}
\renewcommand{\cftchapleader}{\cftdotfill{\cftdotsep}}
\renewcommand{\cftsecleader}{\cftdotfill{\cftdotsep}}
\renewcommand{\cftsubsecleader}{\cftdotfill{\cftdotsep}}
\renewcommand{\cftdotsep}{1}

\pagestyle{empty}

\titleformat{\chapter}[block]{\normalfont\huge\bfseries}{\thechapter}{1em}{}
\titlespacing*{\chapter}{0pt}{-20pt}{10pt}

\titleformat{\section}[block]{\normalfont\Large\bfseries}{\thesection}{1em}{}
\titlespacing*{\section}{0pt}{1ex}{1ex}

\begin{document}

\begin{titlepage}
    \begin{center}
        {\Huge \textbf{Software Libre en Dispositivos Móviles}}\\[0.5cm]
        {\Large \textit{Sistemas operativos, impacto y desarrollo móvil}}\\[1.5cm]

        \includegraphics[width=0.65\textwidth]{portada.png}\\[1cm]

        {\large \textbf{Software Libre y Desarrollo Social}}\\[0.2cm]
        {\large Q1 2025/2026}\\[1cm]

        {\large
        Berta Ferré Segura\\
        Hawjeria Hussain Shaheen\\
        Fardin Arafat Mia\\
        Francisco Ruiz Fernández\\
        Marc Ribas Acon\\
        }

    \end{center}
\end{titlepage}

\tableofcontents
\newpage

%------------------------------APARTADO 1---------------------------------
\pagestyle{plain}
\chapter{Introducción}

{Realizado por: \textbf{Hajweria Hussain}}

En los últimos años, los dispositivos móviles han pasado a formar parte de casi cualquier actividad cotidiana: comunicación, trabajo, educación u ocio. Aun así, no siempre tenemos control real sobre ellos, ya que el ecosistema está dominado por sistemas y servicios cerrados. Aquí es donde el software libre empieza a ser relevante, porque propone un modelo donde el usuario puede recuperar parte del control.

El software libre se basa en cuatro libertades: usar un programa para cualquier propósito, estudiar cómo funciona, modificarlo y compartirlo. Esto lo diferencia del software propietario, que limita estas acciones y mantiene el código cerrado. En movilidad, la mayoría de sistemas son propietarios como iOS, FireOS o HarmonyOS. Android es un caso mixto: su base (AOSP) es libre, pero gran parte del ecosistema comercial usa servicios y capas no libres, como las de Google. Paralelamente, han surgido alternativas libres como LineageOS, /e/OS o postmarketOS, mantenidas por comunidades.

Históricamente, el mercado pasó por una primera etapa variada con plataformas como Symbian o Windows Mobile (2000–2008), hasta concentrarse casi por completo en iOS y Android desde 2010. Esta centralización redujo la diversidad, pero no impidió el crecimiento de proyectos libres alternativos.

No solo hablamos de teléfonos: tablets, UMPCs, MIDs o híbridos como PinePhone o Librem 5 también experimentan con software libre y permiten alargar su vida útil evitando la obsolescencia forzada por fabricantes.

Las ventajas principales del software libre móvil se ven en la privacidad, la transparencia y la posibilidad de mantener dispositivos sin soporte oficial. Sin embargo, hay barreras como bootloaders bloqueados, drivers propietarios y la dependencia de tiendas de apps privativas.

A pesar de esto, proyectos como F-Droid o las ROMs libres demuestran que existen caminos alternativos más éticos, sostenibles y con mayor control para el usuario.

En resumen, el software libre en móviles es posible y socialmente relevante, pero todavía enfrenta limitaciones técnicas y de adopción que deberemos tener en cuenta al analizar el ecosistema actual.


%------------------------------APARTADO 2---------------------------------
\thispagestyle{plain}
\chapter{Sistemas Operativos Libres para Dispositivos Móviles}

{Realizado por: \textbf{Marc Ribas}}

Hoy en dia, el mundo de los Sistemas Operativos para smartphones se trata de un oligopolio entre iOS y Google Android. iOS se describe como un sistema exclusivo, propietario y cerrado que limita la interacción del usuario, la personalización y restringe la instalación de aplicaciones únicamente a las verificadas por Apple, una práctica que la Unión Europea intenta combatir. Por otro lado, Android parte de un proyecto de código abierto al que Google añade una capa de servicios propietarios que gestionan funciones clave, lo cual compromete la privacidad del usuario con fines comerciales. Además, se menciona una futura restricción mediante la Play Integrity API que permitirá bloquear aplicaciones no registradas.

Respecto a las alternativas libres, existen principalmente dos vertientes: los forks del AOSP y las versiones de Linux. GrapheneOS es el ejemplo destacado de los forks, enfocado en la privacidad y seguridad mediante la eliminación de servicios de Google y la implementación de medidas como gestión granular de permisos, aislamiento de aplicaciones en sandboxes, perfiles separados y prevención de exploits. Aunque requiere un dispositivo Google Pixel por su chip TPM para verificar la integridad, soluciona la compatibilidad de aplicaciones permitiendo ejecutar los servicios de Google de manera aislada y sin permisos absolutos.

La otra vertiente busca adaptar GNU/Linux a dispositivos móviles aprovechando la arquitectura ARM y adaptando entornos de escritorio como GNOME o KDE. Para la compatibilidad con aplicaciones existentes, estos sistemas suelen incluir Waydroid, que permite ejecutar apps de Android. Sin embargo, enfrentan el problema de los drivers propietarios de componentes como cámaras y módems, lo que obliga a los desarrolladores a usar capas de compatibilidad o ingeniería inversa. Ejemplos como UbuntuTouch y PostmarketOS ilustran esta categoría, ofreciendo sistemas completamente modificables capaces de revivir hardware antiguo.

En conclusión, aunque los sistemas dominantes son extremadamente convenientes, existen razones de peso como la seguridad y la privacidad para considerar alternativas. Los forks como GrapheneOS ofrecen un equilibrio funcional sin comprometer la privacidad, mientras que los dispositivos Linux, aunque menos desarrollados, permiten una experiencia libre y la recuperación de dispositivos antiguos.

%------------------------------APARTADO 3---------------------------------
\chapter{Aplicaciones de Software Libre para Móviles}

{Realizado por: \textbf{Fardin Arafat}}



El software libre representa una alternativa importante en el ecosistema de dispositivos móviles actual, con especial relevancia en sistemas basados en Android. En este apartado se analiza la disponibilidad de aplicaciones libres en las principales categorías, se comparan con alternativas propietarias y se proporciona la información esencial para comprender este segmento del desarrollo de software móvil.

\section{El Ecosistema de SL Móvil}
Para poder contar con aplicaciones en el móvil es imprescindible disponer de un repositorio o portal de distribución. En el ámbito del software libre, el repositorio más conocido es F-Droid, una alternativa a Google Play Store enfocada en ofrecer únicamente software libre. Sus aplicaciones se compilan directamente desde el código fuente para garantizar que la versión disponible públicamente es exactamente la que se instala, además de asegurar que las dependencias también sean libres. En la actualidad el catálogo supera las 5\,400 aplicaciones.

Otra diferencia clave respecto a Play Store es que F-Droid no requiere cuentas de usuario ni realiza seguimiento alguno: se limita a distribuir software libre. Por ello se ha convertido en un referente para personas preocupadas por la privacidad y la transparencia.

\section{Categorías y Aplicaciones Principales}
En F-Droid las aplicaciones están ordenadas por categorías, lo que facilita cubrir las necesidades del usuario móvil moderno.

	\textbf{Navegación:} Firefox es una referencia dentro de la comunidad del software libre, con una experiencia fluida y soporte para extensiones. Brave Browser, basado en Chromium, ofrece una visión centrada en la privacidad, mientras que el navegador Tor permite conectarse a la red Tor para una navegación casi anónima.

	\textbf{Comunicaciones:} Signal es el estándar de mensajería privada. Desarrollada por Signal Foundation, ofrece cifrado de extremo a extremo con código abierto auditado por la comunidad, soporte para llamadas de voz, videollamadas y chats grupales, sin recopilar datos. Element aprovecha la red Matrix para que el usuario pueda elegir dónde alojar sus datos y sincronizar múltiples plataformas. Briar destaca porque no depende únicamente de Internet: también puede utilizar Bluetooth para crear redes malladas resilientes.

	\textbf{Navegación y mapas:} OsmAnd se apoya en los datos de OpenStreetMap y añade funciones como velocímetro, navegación sin conexión y compatibilidad con Android Auto. En F-Droid se distribuye de forma gratuita, mientras que en Play Store es de pago.

	\textbf{Multimedia y productividad:} VLC actúa como reproductor universal de audio y video, capaz de manejar prácticamente cualquier formato. Fossify Gallery proporciona una galería simple y funcional sin recopilar datos. El cliente de Nextcloud permite sincronizar archivos y fotografías con la nube propia de Nextcloud, incluyendo herramientas de productividad asociadas.

La lista podría continuar con más aplicaciones libres y sus funcionalidades únicas. Además de F-Droid existen alternativas como \textit{Obtainium}, que instala aplicaciones directamente desde GitHub.

\section{Comparativa con Software Propietario}
Las preguntas habituales giran en torno a las ventajas frente al software propietario. En ámbitos como privacidad, seguridad, coste o personalización, el software libre suele ofrecer mejores garantías al usuario. En el terreno del hardware, la situación es más compleja por los estrictos estándares de empresas como Apple y su enfoque de ``walled garden''. Aun así, en Europa se empiezan a aliviar algunas restricciones gracias a regulaciones que obligan a permitir tiendas alternativas con mayor presencia de software libre.

\section{Conclusión}
El software libre en dispositivos móviles es completamente viable. Con el crecimiento constante del catálogo de F-Droid, los usuarios pueden funcionar sin depender del software propietario, recuperar su privacidad y control, y hacer valer sus derechos como personas usuarias y no como meros productos de las corporaciones.

\newpage

%------------------------------APARTADO 4---------------------------------
\chapter{Software Libre en Aplicaciones Móviles: Desarrollo, Costes y Difusión}

{Realizado por: \textbf{Berta Ferré}}

El software libre en el ámbito móvil se presenta como una alternativa sólida a las aplicaciones privativas tradicionales. Su ecosistema se apoya principalmente en repositorios comunitarios como F-Droid, donde todas las aplicaciones se publican de manera transparente, sin rastreadores y sin depender de intermediarios que controlen el proceso de distribución. Frente a tiendas privativas como Google Play o Apple App Store, estas plataformas abiertas permiten un mayor control, más privacidad y un modelo más ético de distribución de software.

En cuanto al desarrollo, crear aplicaciones libres no difiere técnicamente del desarrollo tradicional. En Android se utilizan lenguajes como Kotlin o Java, mientras que en iOS se recurre a Swift, aunque con mayores limitaciones por parte de Apple. Además, frameworks libres y multiplataforma como Flutter, Qt, React Native o Kivy facilitan la creación de apps auditables y adaptables. El software libre destaca especialmente por su colaboración comunitaria, lo que acelera la detección de errores y mejora la seguridad general del proyecto.

Respecto a los costes, las diferencias entre plataformas son muy notables. Publicar en Google Play o en la App Store implica tasas obligatorias y comisiones; esto supone una barrera para pequeños desarrolladores o para proyectos independientes. En cambio, en F-Droid la publicación es completamente gratuita, lo que hace que el software libre sea más accesible y sostenible.

La difusión de aplicaciones libres se apoya en la comunidad: foros, repositorios como GitHub, redes descentralizadas como Mastodon y documentación abierta. Aunque carecen del marketing masivo propio de las grandes empresas, estos proyectos crecen impulsados por el interés cada vez mayor en la privacidad, la ética digital y la transparencia.

En conjunto, el software libre móvil constituye un ecosistema viable, seguro y centrado en el usuario. Representa una alternativa tecnológica sostenible y ética que continúa expandiéndose a medida que aumenta la preocupación por la privacidad y la necesidad de recuperar el control sobre nuestros dispositivos móviles.

%------------------------------APARTADO 5---------------------------------

\chapter{Modelos de Negocio y Seguridad}

{Realizado por: \textbf{Francisco Ruiz}}

\pagestyle{plain}

\section{¿Se puede ganar dinero con software libre?}
El software libre permite generar ingresos a través de modelos alternativos como servicios profesionales, soporte técnico, desarrollo personalizado y donaciones. Aunque no se vende una licencia, la comunidad activa y la participación de los usuarios pueden garantizar la sostenibilidad financiera del proyecto. Plataformas como GitHub Sponsors, Patreon, o Ko-fi permiten a los desarrolladores recibir financiación directa.

\section{Modelos}
Los modelos de negocio más comunes para generar ingresos con software libre son el de las donaciones y mecenazgo, el modelo freemium y los servicios añadidos, y la publicidad ética.

El modelo de donaciones y mecenazgo se ha popularizado, especialmente en el ámbito del software libre móvil. En este modelo, las aplicaciones son gratuitas y de código abierto, pero los usuarios pueden contribuir económicamente al proyecto a través de plataformas como Patreon o GitHub Sponsors. Este modelo permite que los desarrolladores mantengan la libertad del software mientras aseguran la viabilidad financiera del proyecto.

El modelo freemium consiste en ofrecer una versión gratuita del software con características limitadas y una versión premium con funciones adicionales. Este modelo es ideal para aplicaciones que requieren servicios extras como almacenamiento en la nube o características avanzadas. A través de las suscripciones o pagos únicos, se generan ingresos adicionales sin comprometer la libertad del código fuente.

Por otro lado, la publicidad ética se basa en la inserción de anuncios no invasivos que respetan la privacidad del usuario. A diferencia de los modelos tradicionales, no se utiliza el rastreo de datos personales. Los anuncios son estáticos y no interrumpen la experiencia del usuario. Además, muchos proyectos permiten eliminar los anuncios a través de donaciones, asegurando la transparencia y manteniendo un modelo económico sostenible.

En cuanto a la comparación de los modelos, cada uno tiene sus ventajas y desventajas. La publicidad ética genera ingresos constantes pero relativamente bajos, mientras que las donaciones dependen de la participación activa de la comunidad. Por su parte, el modelo freemium ofrece una fuente estable de ingresos, aunque debe equilibrarse con una oferta gratuita suficientemente atractiva para captar usuarios.

\section{Seguridad y Privacidad en Software Libre}
El software libre tiene ventajas notables en términos de seguridad. Al ser código abierto, permite que cualquier persona pueda revisar y auditar el código, lo que facilita la detección de vulnerabilidades. Además, los fallos de seguridad suelen corregirse rápidamente debido a la colaboración activa de la comunidad.

El control sobre los permisos y la transparencia en el acceso a los datos es otra ventaja crucial del software libre. A diferencia del software propietario, que a menudo oculta las funcionalidades que acceden a los datos personales, el software libre ofrece un nivel de confianza más alto debido a su naturaleza abierta.

\section{Conclusión}
El software libre no solo es una opción viable para el desarrollo de aplicaciones, sino que también ofrece varios modelos de negocio que permiten a los desarrolladores generar ingresos sin comprometer la libertad del código. A su vez, proporciona mayores garantías de seguridad y privacidad para los usuarios, lo que lo convierte en una opción cada vez más atractiva tanto para desarrolladores como para usuarios.

\newpage

\newpage

%-------------------------- CONCLUSIONES ---------------------------
\chapter*{Conclusiones}
\addcontentsline{toc}{chapter}{Conclusiones}
\thispagestyle{plain}

El software libre en el mundo móvil no es la opción más usada todavía, pero hemos visto que sí existe, y que tiene ventajas que van más allá de lo técnico. Permite tener más control del dispositivo, más privacidad y alarga la vida de muchos aparatos que, con el software oficial, dejarían de servir demasiado rápido.

También queda claro que los sistemas móviles comerciales funcionan casi como un oligopolio, centrados en comodidad, pero no siempre en dar libertad real al usuario. Android es abierto a nivel base, pero no completamente libre por los servicios de Google, mientras que iOS o FireOS mantienen un control casi total del ecosistema. Frente a esto, proyectos como LineageOS, postmarketOS, F-Droid o GrapheneOS demuestran que hay alternativas más transparentes y éticas, aunque todavía limitadas por drivers y bootloaders bloqueados.

Respecto al desarrollo y la economía, hemos visto que el software libre no impide generar ingresos: simplemente usa otros caminos. Las donaciones, el modelo freemium con servicios opcionales o la publicidad ética sin rastreo son opciones reales que sostienen proyectos móviles. Y, desde el punto de vista de la seguridad, el hecho de poder auditar el código da una confianza adicional que no ofrece el software propietario.

En resumen, el software libre en movilidad es:

\begin{itemize}
  \item Posible y con una comunidad activa detrás.
  \item Más transparente y fiable en privacidad.
  \item Capaz de revivir hardware sin soporte oficial.
  \item Monetizable mediante modelos distintos a licencias.
  \item Generalmente más auditable y adaptable.
\end{itemize}

Aunque todavía hay barreras para su adopción masiva, su papel es relevante socialmente y sigue creciendo, sobre todo en usuarios y desarrolladores preocupados por recuperar el control, proteger datos personales y usar la tecnología de forma más sostenible.


\end{document}
